\documentclass{report}
\usepackage[a4paper, total={7in, 10in}]{geometry}
\usepackage{amssymb}
\title{Logarithms}
\author{Razzak Sir}
\date{December 20}
\begin{document}

\maketitle{}
 $$\textbf{Page} \: 73$$
          

\section{Problem}
Evaluate:
$\log_{a} \left(\frac {\sqrt{140}}{2\sqrt{30}} \right)$+$\log_{a} \left(\frac {3\sqrt{12}}{2\sqrt{27}} \right)$+$\log_{a} \left(\frac {a^{3}\sqrt{b^{2}}}
{b\sqrt{a^{2}}} \right)$

= $\log_{a} \left(\frac {\sqrt{2^{2}\times5\times7}}{2\sqrt{2\times3\times5}} \right)$+$\log_{a} \left(\frac {3\sqrt{2^{2}\times3}}{2\sqrt{3^{3}}} \right)$+$\log_{a} \left(\frac {a^{3}\times b}
{b\times a} \right)$

= $\log_{a} \left(\frac {\sqrt{2^{2}\times5\times7}}{\sqrt{2^{3}\times3\times5}} \right)$+$\log_{a} \left(\frac {\sqrt{2^{2}\times3^{3}}}{\sqrt{2^{2}\times 3^{3}}} \right)$+$\log_{a}\left(a^{2}\right)$

= $\log_{a} \left(\frac {2^{2}\times5\times7}{2^{3}\times3\times5} \right)^ {\frac{1}{2}}$+$\log_{a}\left(1\right)$+$2\log_{a}\left(a\right)$

= $\log_{a} \left(\frac {7}{6}\right)^ {\frac{1}{2}}$+$0$+$2\times1$

= $\frac{1}{2}\left(\log_{a} 7-\log_{a} 6\right)$+$2$

\section{Problem}
Evaluate:
$2\log_{10} 3$ + $3\log_{10} 4$ + $2\log_{10} 5$

= $\log_{10} 3^{2}$+$\log_{10} 4^{3}$ + $\log_{10} 5^{2}$

= $\log_{10} \left(3^{2} \times 4^{3} \times 5^{2} \right)$

= $\log_{10}14400$

= $\log_{10} (120)^{2}$

= $2\log_{10} 120$

  $$\textbf{Page} \: 72$$

   \section{Problem}
let,

Initial principle = $p$

$\therefore$ Compound principle (A) = $p+ p\times 40 \% $

Interest rate (r) = $12\%$

time (n) = ?

\hspace{30cm}
k We Know that,
$$A = p\left(1+r\right)^{n}$$
$$ or,\: p+ p\times 40 \%  = p\left(1+r\right)^{n}$$
$$ or,\: p+ p\times \frac{2}{5}  = p\left(1+12\% \right)^{n}$$
$$ or,\: \frac{5p+2p}{5}  = p\left(1+12\% \right)^{n}$$
$$ or,\: \frac{7p}{5}  = p\left(1+0.12 \right)^{n}$$
$$ or,\: \frac{7}{5}  = \left(1.12 \right)^{n}$$
$$ or,\: \log{\frac{7}{5}}  = \log{\left(1.12 \right)}^{n}$$
$$ or,\: \log{\frac{7}{5}}  = n\log{\left(1.12 \right)}$$
$$ or,\: n  = \frac{\log{\frac{7}{5}}}{log{\left(1.12 \right)}}$$
$$\therefore \: n \approx 3$$

\hspace{30cm}
$$\textbf{Page} \: 73$$
\section{Problem}
let,

Initial price =$P$

$\therefore$ After $5$ years, decreased price $(P_T) = P-P\times60\%$

\hspace{30cm}
1 We know that,
$$P_T = P\left(1-R\right)^T$$
$$ or,\:\frac{P}{2} = P(1-R)^{5}$$
$$ or,\:\frac{1}{2} = (1-R)^{5}$$
$$ or,\:\left(\frac{1}{2}\right)^\frac{1}{5} = (1-R)$$
$$ or,\:R = 1-\left(\frac{1}{2}\right)^{\frac{1}{5}}$$
$$\therefore R = 0.129$$

As initial price =$P$

$\therefore$ After $n$ years, the decreased price will be = $P-P\times60\%$

\hspace{30cm}
1 According to the question,
$$P-P\times60\% = P(1-0.129)^{n}$$
$$or,\:P-\frac{3P}{5} = P(0.871)^{n}$$
$$or,\:\frac{2P}{5} = P(0.871)^{n}$$
$$or,\:\frac{2}{5} = (0.871)^{n}$$
$$or,\:(0.871)^{n} = \frac{2}{5}$$
$$or,\:n = \log_{0.871}\frac{2}{5}$$
$$\therefore n \approx 6.63 $$

\hspace{30cm}
$$\textbf{Page} \: 75$$
\section{Problem}
let,

Intensity of the first earthquake  = $I_5$

Intensity of the second earthquake = $I_7$ 

Intensity of an ideal earthquake = $S$

\hspace{30cm}

$\therefore$ The Richter magnitude of the first earthquake is $\log_{10}\left(\frac{I_5}{S}\right)$ = 5 ... (i)

$\therefore$ The Richter magnitude of the first earthquake is $\log_{10}\left(\frac{I_7}{S}\right)$ = 7 ... (ii)

\hspace{30cm}
k (ii)-(i)

$$\log_{10}\left(\frac{I_7}{S}\right) - \log_{10}\left(\frac{I_5}{S}\right) = 7-5$$
$$or,\: \log_{10} \frac{\left(\frac{I_7}{S}\right)}{\left(\frac{I_5}{S}\right)} = 2$$
$$or,\: \log_{10}\left(\frac{I_7}{S}\times\frac{S}{I_5}\right) = 2 $$
$$or,\: \log_{10}\left(\frac{I_7}{I_5}\right) = 2$$
$$or,\: \frac{I_7}{I_5} = 10^{2}$$
$$or,\: \frac{I_7}{I_5} = 100$$
$$\therefore I_7 = 100 \times I_5$$


Again,


Intensity of the first earthquake  = $I_5$

Intensity of the second earthquake = $I_8$ 

Intensity of an ideal earthquake = $S$

\hspace{30cm}

$\therefore$ The Richter magnitude of the first earthquake is $\log_{10}\left(\frac{I_5}{S}\right)$ = 5 ... (i)

$\therefore$ The Richter magnitude of the first earthquake is $\log_{10}\left(\frac{I_8}{S}\right)$ = 8 ... (ii)

\hspace{30cm}
k (ii)-(i)

$$\log_{10}\left(\frac{I_8}{S}\right) - \log_{10}\left(\frac{I_5}{S}\right) = 8-5$$
$$or,\: \log_{10} \frac{\left(\frac{I_8}{S}\right)}{\left(\frac{I_5}{S}\right)} = 3$$
$$or,\: \log_{10}\left(\frac{I_8}{S}\times\frac{S}{I_5}\right) = 3 $$
$$or,\: \log_{10}\left(\frac{I_8}{I_5}\right) = 3$$
$$or,\: \frac{I_8}{I_5} = 10^{3}$$
$$or,\: \frac{I_8}{I_5} = 1000$$
$$\therefore I_8 = 1000 \times I_5$$



(Showed)


$$\textbf{Page} \: 76$$
\section{Problem}
let,

Intensity of the earthquake measured in Manikganj = $I_1$

Intensity of the earthquake measured in Rangamati = $I_2$ 

Intensity of an ideal earthquake = $S$

\hspace{30cm}

$\therefore$ The Richter magnitude of the measured in Manikganj is $\log_{10}\left(\frac{I_1}{S}\right)$ = 7.0 ... (i)

$\therefore$ The Richter magnitude of the earthquake measured in Rangamati is $\log_{10}\left(\frac{I_2}{S}\right)$ = 5.1 ... (ii)

\hspace{30cm}
k (i)-(ii)

$$\log_{10}\left(\frac{I_1}{S}\right) - \log_{10}\left(\frac{I_2}{S}\right) = 7.0-5.1$$
$$or,\: \log_{10} \frac{\left(\frac{I_1}{S}\right)}{\left(\frac{I_2}{S}\right)} = 1.9$$
$$or,\: \log_{10}\left(\frac{I_1}{S}\times\frac{S}{I_2}\right) = 1.9 $$
$$or,\: \log_{10}\left(\frac{I_1}{I_2}\right) = 1.9$$
$$or,\: \frac{I_7}{I_5} = 10^{1.9}$$
$$or,\: \frac{I_1}{I_2} = 79.43$$
$$or,\: \frac{I_1}{I_2} \approx 80$$
$$\therefore I_1 \approx 80 \times I_2$$

$$\textbf{Page} \: 78$$

\section{Problem}
We know that 

Sound level is = $d\log_{10})\left(\frac{I}{S}\right)$

here,

$I = 2.35\times10^{-6} w/m^{2}$

$S = 10^{-12} w/m^{2}$

\hspace{30cm}

$\therefore d =10\log_{10}\left(\frac{2.35\times10^{-6}w/m^{2}}{10^{-12}w/m^{2}}\right)$

$\quad\quad =10\log_{10}\left(\frac{2.35\times10^{-6}}{10^{-12}}\right)$

$\quad\quad =10\log_{10} \left(2.35\times10^{6}\right)$

$\quad\quad = 10 \times 6.371$

$\quad\quad = 63.71$

$\quad\quad \approx 64$

$\therefore$ The sound level is approximately 64 decibel

$$\textbf{Page} \: 80$$

\section{Problem}
\subsection{}
Evaluate: $2\sqrt[3]{343}+2\sqrt[5]{243}-12\sqrt[6]{64}$

$\quad\quad =2\sqrt[3]{7^3}+2\sqrt[5]{3^5}-12\sqrt[6]{2^6}$

$\quad\quad =2\times7+2\times3+2\times2$

$\quad\quad = 24$

\subsection{}
Evaluate: $\frac{y^{a+b}}{y^{2c}}\times\frac{y^{b+c}}{y^{2a}}\times\frac{y^{c+a}}{y^{2b}}$

$\quad\quad = y^{a+b-2c}\times y^{b+c-2a}\times y^{c+a-2b}$

$\quad\quad = y^{a+b-2c+b+c-2a+c+a-2b}$

$\quad\quad = y^{0}$

$\quad\quad = 1$

\section{Problem}

Evaluate: $\left(\frac{z^a}{z^b}\right)^{a+b-c}\times\left(\frac{z^b}{z^c}\right)^{b+c-a}\times\left(\frac{z^a}{z^a}\right)^{c+a-b}$

$\quad\quad = z^{(a-b)(a+b-c)}\times z^{(b-c)(b+c-a)}\times z^{(c-a)(c+a-b)}$

$\quad\quad = z^{a^2-b^2-ac+bc}\times z^{b^2-c^2-ab+ac}\times z^{c^2-a^2-bc+ab}$

$\quad\quad =z^{a^2-b^2-ac+bc+b^2-c^2-ab+ac+c^2-a^2-bc+ab}$

$\quad\quad =z^{0}$

$\quad\quad = 1$

\section{Problem}
\subsection{}
$\quad\quad\quad2^x = 64$

or, $\log 2^x = \log64$

or, $x\log 2 = \log 2^{6}$

or, $x = \frac{6\log2}{\log2}$

$\therefore$ $x =6$

\subsection{}
$\quad\quad\quad\left(1.2\right)^x = 100$

or, $\log \left(1.2\right)^x = \log100$

or, $x\log \left(1.2\right) = \log 10^2$

or, $x = \frac{2\log10}{\log\left(1.2\right)}$

$\therefore$ $x =25.259$

\subsection{}
$\quad\quad\quad7^x = 5$

or, $\log 7^x = \log5$

or, $x\log 7 = \log 5$

or, $x = \frac{log5}{\log7}$

$\therefore$ $x =0.827$

\subsection{}
$\quad\quad\quad\left(\frac{2}{3}\right)^x = 7$

or, $\log \left(\frac{2}{3}\right)^x = \log7$

or, $x\log \left(\frac{2}{3}\right) = \log 7$

or, $x = \frac{\log7}{\log\left(\frac{2}{3}\right)}$

$\therefore$ $x =-4.8$

\section{Problem}

let,

Initial principle = $p$

$\therefore$ Compound principle (A) = $3p $

Interest rate (r) = $10\%$

time (n) = ?

\hspace{30cm}
k We Know that,
$$A = p\left(1+r\right)^{n}$$
$$ or,\: 3p = p\left(1+10\%\right)^{n}$$
$$ or,\: 3  = \left(1+0.1 \right)^{n}$$
$$ or,\: \log{3}  = \log{\left(1.1 \right)}^{n}$$
$$ or,\: \log{3}  = n\log{\left(1.1 \right)}$$
$$ or,\: n  = \frac{\log{3}}{log{\left(1.1 \right)}}$$
$$\therefore \: n \approx 11.53 $$

\section{Problem}

\: \quad After 1 day the number of affected people will be = $3^{1}$

After 2 days the number of affected people will be = $3^{2}$

After 3 days the number of affected people will be = $3^{3}$

$\therefore$ After 30 days the number of affected people will be = $3^{30}$

\quad\quad\quad\quad\quad\quad\quad\quad\quad\quad\quad\quad\quad\quad\quad\quad\quad\quad\quad\quad\quad\quad\qquad $= 2.0589 \times 10^14$


\hspace{35cm}
1\quad\ \hspace{3cm} \quad After 1 day the number of affected people will be = $3^{1}$

$\therefore$ After n days the number of affected people will be = $3^{n}$

\hspace{35cm}
1 According to the question,

$$3^{n} = 10^{7}$$
$$or,\:\log 3^{n} = \log 10^{7}$$
$$or, \: n\log 3 = 7\log 10 $$
$$or, \: n = \frac{7\log 10}{log 3} $$
$$\therefore \: n = 14.67 $$

\section{Problem}

We know that ,

1 Bigha = $20 $ Katha

$\therefore$ 3 Bigha = $20\times 3$ Katha

\:\:\:\:\quad\quad\quad\quad = $60$ Katha

\hspace{35cm}

$1$ kg fertilizer  increase the fertility by = $3\%$

$\therefore$ $30$ kg fertilizer increase the fertility by = $30\times3\%$

\hspace{6cm} = $90\%$

Given that,

The amount of fertile land $(P)$ = $60$ Katha

The fertility reduction rate $(R)$ = $90\%$

time $(n)$ = $1$

We know that,

$$Depriciation\quad (P_T)= P(1-R)^{n} $$
$$or,\: (P_T) = 60(1-90\%)^{1}$$
$$ = 60\times(1-\frac{90}{100})^{1}$$
$$ = 60\times \frac{1}{10}$$
$$= 6$$

$\therefore$ The amount of time it would take the land to lose it's fertility is = $\frac{60}{6}$ years
$\ = 10  $ years

\section{Problem}

let,

Intensity of the earthquake measured in Sreemangal = $I_1$

Intensity of the earthquake measured in Chattogram = $I_2$ 

Intensity of an ideal earthquake = $S$

\hspace{30cm}

$\therefore$ The Richter magnitude of the earthquake measured in Sreemangal is $\log_{10}\left(\frac{I_1}{s}\right)$ = 7.6 ... (i)

$\therefore$ The Richter magnitude of the earthquake measured in Chattogram is $\log_{10}\left(\frac{I_2}{s}\right)$ = 6.0 ... (ii)

\hspace{30cm}
k (i)-(ii)

$$\log_{10}\left(\frac{I_1}{s}\right) - \log_{10}\left(\frac{I_2}{s}\right) = 7.6-6.0$$
$$or,\: \log_{10} \frac{\left(\frac{I_1}{S}\right)}{\left(\frac{I_2}{S}\right)} = 1.6$$
$$or,\: \log_{10}\left(\frac{I_1}{S}\times\frac{S}{I_2}\right) = 1.6 $$
$$or,\: \log_{10}\left(\frac{I_1}{I_2}\right) = 1.6$$
$$or,\: \frac{I_7}{I_5} = 10^{1.6}$$
$$or,\: \frac{I_1}{I_2} = 39.81$$
$$or,\: \frac{I_1}{I_2} \approx 40$$
$$\therefore I_1 \approx 40 \times I_2$$

\section{Problem}

let,

Intensity of the first earthquake = $I_1$

$\therefore$ Intensity of the second earthquake  = $6I_1$ 

Intensity of an ideal earthquake = $S$

\hspace{30cm}

$\therefore$ The Richter magnitude of the first earthquake is = $\log_{10}\left(\frac{I_1}{S}\right)$

$\therefore$ The Richter magnitude of the second earthquake is = $\log_{10}\left(\frac{6I_1}{S}\right)$

According to the question,

$$\log_{10}\left(\frac{I_1}{S}\right) = 8 $$
$$or, \: \frac{I_1}{S} = 10^{8} $$
$$or, \: \frac{6I_1}{S} = 6\times10^{8} $$
$$or, \: \log\left(\frac{6I_1}{S}\right) = \log(6\times10^{8}) $$
$$or, \: \log\left(\frac{6I_1}{S}\right) = 8.78 $$

$\therefore$  The Richter magnitude of the second earthquake is 8.78

\section{Problem}

let,

Intensity of the earthquake measured in Cox's Bazar = $I_1$

Intensity of the earthquake measured in Turkey = $398I_1$ 

Intensity of an ideal earthquake = $S$

\hspace{30cm}

$\therefore$ The Richter magnitude of earthquake measured in Cox's Bazar is = $\log_{10}\left(\frac{I_1}{S}\right)$

$\therefore$ The Richter magnitude of earthquake measured in Turkey is = $\log_{10}\left(\frac{398I_1}{S}\right)$

According to the question,

$$\log_{10}\left(\frac{I_1}{S}\right) = 5.2 $$
$$or, \: \frac{I_1}{S} = 10^{5.2} $$
$$or, \: \frac{398I_1}{S} = 398\times10^{5.2} $$
$$or, \: \log\left(\frac{398I_1}{S}\right) = \log(398\times10^{5.2}) $$
$$or, \: \log\left(\frac{6I_1}{S}\right) = 7.8 $$

$\therefore$  The Richter magnitude of earthquake measured in Turkey is  7.8








\end{document}
